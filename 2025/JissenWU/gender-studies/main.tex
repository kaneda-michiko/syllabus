\documentclass{jlreq}
\usepackage[left=2cm,right=2cm,top=2cm,bottom=2cm]{geometry}
\usepackage{multicol}
\usepackage{array}
\usepackage{longtable}
\usepackage{graphicx}
\usepackage{hyperref}
\usepackage{titlesec}

% Title formatting
\titleformat{\section}{\large\bfseries}{}{0em}{}
\titleformat{\subsection}{\normalsize\bfseries}{}{0em}{}

% Header formatting
\renewcommand{\arraystretch}{1.5}

\begin{document}
\pagenumbering{gobble}

% Title
\begin{center}
    \huge{ジェンダーと私/私たち} \\
    \large{\textbf{フェミニズム理論におけるジェンダー・アイデンティティ}} \\
    \vspace{0.3cm}
    \large{担当講師: 金田迪子 \quad 学期: 前期}
\end{center}
\vspace{0.5cm}

\begin{multicols}{2}
% Course Theme
\section*{授業のテーマ}
自分について、「私」という人称を使うことに違和感を感じたことはありますか。女性でよかったなと思うとき、あるいは、女性でなければよかったのになと思うときはありますか。この授業では女性という主体(subject)について学びます。専門的には、フェミニズム理論と、そこから批判的に派生した諸理論の流れを扱います。とはいえ、多くのみなさんにとっては初めて読むタイプの本となりますので、「理論」と呼ばれてきた本に親しむこともこの授業のテーマとします。
% Course Objectives
\section*{授業における到達目標}
主体としての女性をめぐるさまざまな考えを学ぶことによって、ジェンダーやセクシュアリティの多様性に触れ、国際的視野を養います。また、西洋哲学などの近接領域で、主体と女性の問題について用いられてきた様々な表現に原語で触れることにより、美の探究を行います。

% Weekly Plan Table
\section*{授業の内容}
\begin{itemize}
\item 第1回 授業への参加方法
\item 第2回 ページェント
\item 第3回 シェイクスピア劇
\item 第4回 レストレード劇場
\item 第5回 ヴィクトリア朝劇場
\item 第6回 モダン劇場
\item 第7回 ポストモダン劇場
\item 第8回 パフォーマンスアート
\item 第9回 ダンス
\item 第10回 ミュージカル
\item 第11回 オペラ
\item 第12回 バレエ
\item 第13回 パントマイム
\item 第14回 まとめ(定期試験対策)
\end{itemize}

% Independent Study Outside of Class
\section*{事前・事後学習}
この授業はオンデマンド形式の講義授業ですが、単位の修得のためには時間外学習が必要です。以下を本授業の時間外学習として課します。
\begin{itemize}
    \item 手元に研究ノートを用意する(Notionなどがおすすめ)。
    \item 次の授業の概要を確認し、PDFを手元に用意する。気になることをノートに書く。特に自分にとって関心のある内容の回は、PDFの内容に目を通してメモをとっておく。
    \item 授業を受けた後、ノートを見直す。期末試験の要点などを、ノートを加筆する。
\end{itemize}

% Term-End Exam
\section*{定期試験}
実施

% Assessment and Feedback
\section*{成績評価の方法・基準とフィードバック}
\begin{itemize}
    \item 平常点(レスポンスペーパー) 50%
    \item 期末試験 50%
\end{itemize}

% References
\section*{教科書・教材等}
manaba上でPDFを配布します。

% References
\section*{参考書}
manaba上で主要参考文献を提示します。

% Working Experience
\section*{授業に活かす実務経験}
なし

\section*{アクティブラーニング}
実施しない

\section*{その他}
特記事項なし

\end{multicols}
\end{document}
